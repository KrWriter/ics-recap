% Copyright 2019 Clara Eleonore Pavillet

% Author: Clara Eleonore Pavillet (original author), Leonard Quentin Marcq (modified for tsinghua), 赵宗义 (参与了修改), Li(modified for PKU)
% Description: This is an unofficial Peking University Beamer Template I made from scratch. Feel free to use it, modify it, share it.
% Repo Address: https://github.com/synthpop123/PKU-Beamer-Template
% Version: 1.1.1


\documentclass{beamer}
\usepackage{ctex} % for Chinese support
\usepackage{fontspec}
\usepackage{listings}
\usepackage{tikz}
\input{Theme/Packages.tex}
\usetheme{pku}

%%%<<<---
\newcommand{\topic}{MallocLab Review}
\newcommand{\variable}{\mathrm}
\newcommand{\field}{\mathsf}
%%%--->>>

\title{\textbf{\topic}}
\titlegraphic{\includegraphics[width=2cm]{Theme/Logos/pku_emblem.png}}
\author{Yuxuan Kuang}
\institute{School of EECS, PKU}
\date{2022-12-07}

\begin{document}

{\setbeamertemplate{footline}{} 
\frame{\titlepage}}

\begin{frame}{Outline}
\begin{enumerate}
	\item 思路
	\item 实现
	\item Tricks
\end{enumerate}
\end{frame}

\begin{frame}{思路}
\begin{enumerate}
	\item 隐式空闲链表
	\item 显式空闲链表
		\begin{enumerate}
			\item 简单分离存储
			\item 分离适配
			\item 伙伴系统
		\end{enumerate}
\end{enumerate}
\end{frame}

\begin{frame}{实现}
\begin{enumerate}
	\item \texttt{mm\_init}, \texttt{malloc}, \texttt{free}, \dots
	\item helper functions
		\begin{enumerate}
			\item \texttt{extend\_heap}
			\item \texttt{coalesce}
			\item \texttt{place}
			\item \dots
		\end{enumerate}
	\item \texttt{mm\_checkheap}
\end{enumerate}
\end{frame}

\begin{frame}{实现-1}
	\texttt{mm\_init}, \texttt{malloc}, \texttt{free}, \dots
\end{frame}

\begin{frame}{实现-2}
	helper functions
	\begin{enumerate}
		\item \texttt{extend\_heap}
		\item \texttt{coalesce}
		\item \texttt{place}
		\item \dots
	\end{enumerate}
\end{frame}

\begin{frame}{实现-3}
	\texttt{mm\_checkheap}
\end{frame}

\begin{frame}{Tricks}
	\begin{enumerate}
		\item 去脚部
		\item 压指针
		\item BST
		\item \dots
	\end{enumerate}
\end{frame}
  
\begin{frame}{}
\begin{center}
	\textbf{Thank you for your attention!}
\end{center}
\end{frame}


\end{document}

